%%%%%%預先準備(不需修改)%%%%%%
\documentclass[12pt]{article}
\input{preamble}
\usepackage{graphicx}

%%%%%%%%自訂設置(請填入)%%%%%%%%
\centerhead{血糖控制與糖尿病照顧} %中上方頁首,請填入小論文主題名稱

%%%%%%%%文件內容開始%%%%%%%%%%%%
\begin{document}
\coverpage
    {生物類} %eg.工程技術類
    {血糖控制與糖尿病照顧} %填入主題名稱
    %填入作者。若作者只有 1 位 或 2 位,則將 {} 內留白即可
    {28 林珈生。國立政治大學附屬高級中學。高一 4 班} %第一作者
    {} %第二作者
    {} %第三作者
    {林以荷老師} %指導老師
    
\h{前言}
\hh {研究動機}
\hhtext{
    我大概在小學五年級的暑假被診斷出第一型糖尿病,當時對於血糖的控制是迷迷糊糊,在經過多年的經驗累積,現在可以以較精準的判斷控制血糖。因影響血糖的因素有太多種,從最基本的飲食、胰島素劑量、活動量,到比較無法控制的心情起伏、賀爾蒙、氣溫等等,而難以控制,所以我打算來了解「放假與平日」的血糖差異以及「季節變化對血糖的影響」。
    }
\hh {研究項目}
    \hhh{假日及平日的血糖波動差異}
    \hhh{季節對血糖的影響}
\h{文獻探討}
\hh{研究方向}
\hhtext{
過去判定血糖的方式都是透過糖化血色素(HbA1c),但因只能看到平均,其實對於高低血糖的表現不是很清楚,就是有時很高,有時很低,但最後出來的HbA1c剛好在標準內。而現在醫師通常都會看TIR(Time In Range, 血糖安全範圍),來判斷控制是否需調整。所以這次的實驗結果判斷除了比較之外,會依據醫生給的標準(80~200 mg/dl),來判定。而如果再進一步的要求,會想要整體曲線是平穩的,如有忽高忽低的狀況(如圖二,俗稱坐雲霄飛車),也不是一個好結果。
}

\begin{figure}[H] %H代表強制圖片放在原始碼對應的相關位置,htbp代表由 LaTeX 彈性調整
\captionsetup{format=hang, singlelinecheck=off}
\centering %圖片置中
\vspace{0pt}
\begin{minipage}[t]{0.75\textwidth}
\caption{黑線為最高值,白線為最低值,理想血糖在這個區間內}
\centering %圖片在 minipage 置中\
\includegraphics[width=0.85\textwidth]{Images//ideal glucose fluctuation.png}\\[12pt]
資料來源:研究者自行繪製
\label{fig.1}
\end{minipage}
\end{figure}


\begin{figure}[H] %H代表強制圖片放在原始碼對應的相關位置,htbp代表由 LaTeX 彈性調整
\captionsetup{format=hang, singlelinecheck=off}
\centering %圖片置中
\vspace{0pt}
\begin{minipage}[t]{0.75\textwidth}
\caption{血糖值忽高忽低}
\centering %圖片在 minipage 置中\
\includegraphics[width=0.85\textwidth]{Images//roller coaster Glucose.png}\\[12pt]
資料來源:研究者自行繪製
\label{fig.2}
\end{minipage}
\end{figure}






\h{研究方法}
    \hh{工具}
    \hhh{連續血糖監測}
    \hhhtext{
    連續血糖監測(CGM continue glucose monitoring),是將像貝殼的感應器長期貼在身體上,透過探針接收組織液中的葡萄糖濃度,來達到連續檢測。通常需要幾次的的指尖血校正,數值才會準確。 
}

    \hhh{自行編譯的程式}
    \hhhtext{
    透過Python編譯的資料擷取程式,讓資料擷取時間縮短。主要是幫忙擷取目標時間的資料,再匯出平均。
}
    \begin{figure}[htbp]
    \centering
    \begin{minipage}[t]{0.5\textwidth}
    \centering
    \includegraphics[width=8cm, height=5cm]{Images/Program.png}
    \caption{}
    \end{minipage}
    \begin{minipage}[t]{0.49\textwidth}
    \centering
    \includegraphics[width=8cm, height=5cm]{Images/Program 2.png}
    \caption{}
    資料來源: 研究者自行繪製
    \end{minipage}
    \end{figure}
    
\clearpage


    \hh{研究流程}
    \hhtext{
    本次統計將透過作者兩年紀錄下來總共五萬多筆資料,比較假日與平日的血糖差異與季節對於血糖的影響。
    }      
        
        
    \begin{figure}[H] %H代表強制圖片放在原始碼對應的相關位置,htbp代表由 LaTeX 彈性調整
    \captionsetup{format=hang, singlelinecheck=off}
    \centering %圖片置中
    \vspace{0pt}
    \begin{minipage}[t]{0.85\textwidth}
    \caption{研究流程}
    \centering %圖片在 minipage 置中\
    \includegraphics[width=0.95\textwidth]{Images//Biology Process.png}\\[12pt]
    資料來源:研究者自行繪製
    \label{fig.5}
    \end{minipage}
    \end{figure}





    \h{研究分析與結果}
    \hh{假日與平日的血糖差異}
    \hhtext{
    透過城市與Excel分析後,得到的解果
    
    
    }
    \begin{figure}[H] %H代表強制圖片放在原始碼對應的相關位置,htbp代表由 LaTeX 彈性調整
    \captionsetup{format=hang, singlelinecheck=off}
    \centering %圖片置中
    \vspace{0pt}
    \begin{minipage}[t]{0.9\textwidth}
    \caption{假日與平日的血糖差異}
    \centering %圖片在 minipage 置中\
    \includegraphics[width=0.95\textwidth]{Images//weekdays and Holidays.png}\\[12pt]
    資料來源:研究者自行繪製
    \label{fig.6}
    \end{minipage}
    \end{figure}
    
    
    
    \hhtext{
    
    
    }
    
    \h{研究結論與建議}

\clearpage %後方的內容獨立從新的一頁開始
\h{參考文獻}

%%%%%%%%%%%%%%%%%%%%%%%%%%%%%%%%%%%%%%%%%%%%%
%此區塊為參考文獻,請直接編輯 ref.tex 檔案%
\begin{enumerate}[(1)]
\input{ref}
\end{enumerate}
%%%%%%%%%%%%%%%%%%%%%%%%%%%%%%%%%%%%%%%%%%%%%

\end{document}
